%------------------------------------------------------------------------------
\chapter{Uma Breve Revis�o de Modelagem e Previs�o de S�ries Temporais}

%\vspace{-1cm} \label{cap2}
%
%\begin{flushright}
%\begin{minipage}{0.7\linewidth}
%\emph{``Os f�sicos acham que tudo o que temos de fazer � dizer:
%estas s�o as condi��es, o que acontece em seguida?''}
%\end{minipage}
%\end{flushright}
%
%\begin{flushright}
%{Richard Phillips Feynman (1918-1988)}
%\end{flushright}
%
%%\vspace{1cm}
%
%%Constam nesse cap�tulo a revis�o bibliogr�fica das t�cnicas de
%%previs�o de consumo e demanda energ�tica a longo prazo, bem como dos
%%algoritmos utilizados para a an�lise de dados sub-rogados.
%
%
%\section{Modelagem e Previs�o de S�ries Temporais}
%
%\section{Modelagem e Previs�o de S�ries Temporais de Consumo de Energia El�trica}
%\markright{\thesection ~~~ Modelos de Previs�o} \label{previsao}
%
%\clearpage
